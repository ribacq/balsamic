% a simple blog design — Quentin Ribac © 2017
% This is free software; see LICENSE file for more information.
\documentclass[a4paper, 11pt]{article}
\usepackage{jipkg}
\usepackage{tikz-uml}

\jihypersetup{A simple blog design}{Quentin Ribac}
\title{A simple blog design}
\author{Quentin \textsc{Ribac}}
\date{\today}

\begin{document}
\selectlanguage{english}
\maketitle

\section{Class Diagram}
\begin{figure}[h]\label{pkg: blog}\center{
\begin{tikzpicture}
	\begin{umlpackage}{blog}
		% Classes
		\umlclass[x = 6, y = -9]{User}{
			isAdmin : bool
		}{}
		\umlclass[x = 0, y = -3]{Comment}{
			body : string\\
			createdAt : date\\
			editedAt : date\\
			inTrash : bool
		}{}
		\umlclass[x = 6, y = -3]{Article}{
			title : string(140)\\
			body : string\\
			createdAt : date\\
			editedAt : date\\
			isListed : bool\\
			isDraft : bool\\
			inTrash : bool
		}{}
		\umlnote[x = 1, y = 0, width = 5.5cm]{Article}{An unlisted article is still readable.\\A draft is unlisted and unreadable.}
		\umlclass[x = 12, y = 0]{Category}{
			name : string(140)\\
			description : string(420)
		}{}
		\umlclass[x = 12, y = -3]{Series}{
			name : string(140)\\
			description : string(420)
		}{}
		\umlclass[x = 12, y = -6]{Tag}{
			name: string(42)
		}{}

		% Relations
		\umluniassoc[mult1 = *, mult2 = 1, anchor1 = 0, name = comArt]{Comment}{Article} % Comment answers Article
		\umluniassoc[mult1 = *, mult2 = 1, angle1 = -20, angle2 = -60, loopsize = 2cm, name = comCom]{Comment}{Comment} % Comment answers Comment
		\umlassoc[stereo = XOR]{comArt-1}{comCom-1} % Comment answers Article XOR Comment

		\umluniassoc[mult1 = *, attr2 = authors|1..*]{Article}{User} % Article has 1+ authors
		\umlVHuniassoc[mult1 = *, attr2 = author|1]{Comment}{User} % Comment has 1 authors

		\umluniassoc[mult1 = *, mult2 = 1]{Article}{Category} % Article is in 1 Category
		\umluniassoc[mult1 = *, mult2 = 0..1]{Article}{Series} % Article is in 0 or 1 Series
		\umluniassoc[mult1 = 1..*, mult2 = *]{Article}{Tag} % Article has 0+ Tag
	\end{umlpackage}
\end{tikzpicture}
\caption{Packgage: \texttt{blog}}
}\end{figure}

\section{List of Meteor templates}
\subsection{Components}
\begin{description}
	\item[article-body] shows article body
	\item[article-meta] shows article metadata (authors, category, series, etc.)
	\item[article-short] shows article for display in feed (small excerpt from the beginning)
	\item[article-edit] a simple form
	\item[comment] shows comment
	\item[comment-edit] a simple form
	\item[category] shows name, description and number of member articles
	\item[category-edit] a simple form
	\item[series] shows name, description, number of member articles
	\item[series-edit] a simple form
	\item[user-rtfm] manages user login, logout and register with forms
	\item[user-status] shows name of connected user, links to profile
\end{description}

\subsection{Layouts}
\begin{description}
	\item[main] with sidebar
	\item[plain] without sidebar
\end{description}

\subsection{Pages}
\begin{description}
	\item[feed] serves as home, category and series page; uses main-layout
	\item[read] shows a single article with its comments tree; uses main-layout
	\item[user] shows user profile public information; uses plain-layout
	\item[user-edit] a not-so-simple form; uses plain-layout
	\item[admin] lets you edit blog title and such global variables; uses plain-layout
\end{description}

\section{Sidebar contents}
\begin{enumerate}
	\item research bar
	\item user-status-component and user-login-logout-component
	\item if connected user is an admin, link to create a new article
	\item current category if one; uses category-component
	\item available categories as a list plus “all”
	\item current series if one; uses series-component
	\item enabled tags
	\item available tags
	\item misc. filters (author, date)
	\item article-meta-component if applicable
	\item about blog: a special article
\end{enumerate}

\end{document}
